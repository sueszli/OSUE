\input{../../../template.ltx}

\begin{document}

\osuetitle{1}

%\section*{Aufgabenstellung A -- ispalindrom}
\section*{Assignment A -- ispalindrom}

%Implementieren Sie ein Programm \osueprog{ispalindrom}, welches Strings auf
%Palindromeigenschaften überprüft.
Implement a program \osueprog{ispalindrom},
which checks whether strings are palindroms.

\begin{verbatim}
    SYNOPSIS:
        ispalindrom [-s] [-i] [-o outfile] [file...]
\end{verbatim}

%Das Programm \osueprog{ispalindrom} soll zeilenweise Strings mit einer Länge
%von bis zu 40 echten Zeichen lesen; prüfen, ob ein Palindrom vorliegt, d.h.\ ob
%der Text rückwärts gelesen mit sich selbst ident ist; und den Text gefolgt
%von \osueoutput{ist ein Palindrom} bzw. \osueoutput{ist kein Palindrom}
%ausgeben.
The program \osueprog{ispalindrom} shall read files line by line
and for each line check whether it is a palindrom,
i.e. whether the text read backwards is identical to itself.
Each line shall be printed followed by the text
\osueoutput{``is a palindrom"} if the line is a palindrom
or \osueoutput{``is not a palindrom"} if the line is a not palindrom.

Your program must accept lines of any length.

%Wird keine Eingabedatei (\verb|infile|) angegeben, so ist von \osueglvar{stdin}
%zu lesen. Fehlt die Option \verb|-o|, wird auf \osueglvar{stdout} ausgegeben.
If one or multiple input files are specified (given as positional arguments),
then \osueprog{ispalindrom} shall read each of them in the order they are given.
If no input file is specified, the program reads from \osueglvar{stdin}.

If the option \osuearg{-o} is given,
the output is written to the specified file (\verb|outfile|).
Otherwise, the output is written to \osueglvar{stdout}.

%Die Option \osuearg{-s} soll bewirken, dass Leerzeichen ignoriert werden; die
%Option \osuearg{-i} soll bewirken, dass nicht zwischen Groß- und
%Kleinschreibung unterschieden wird.
The option \osuearg{-s} causes the program to ignore whitespaces
when checking whether a line is a palindrom.
If the option \osuearg{-i} is given,
the program shall not differentiate between lower and upper case letters,
i.e. the check for a palindrom shall be case insensitive.

\paragraph*{Hint:}
The functions \verb|tolower(3)| or \verb|toupper(3)|
might be useful when implementing the case insensitive check.

%\subsection*{Testen}
\subsection*{Testing}

%Testen Sie Ihr Programm mit verschiedenen Eingaben, wie z.B.:
Test your program with various inputs, such as:

\begin{osuefmtcode}
      $ \osueinput{./ispalindrom}
      \osueinput{Racecar}
      Racecar is not a palindrom
      \osueinput{racecar}
      racecar is a palindrom

      $ \osueinput{./ispalindrom -i -s}
      \osueinput{Racecar}
      Racecar is a palindrom
      \osueinput{Was it a car or a cat I saw}
      Was it a car or a cat I saw is a palindrom

      $ \osueinput{cat example.in}
      Madam
      never odd or even
      $ \osueinput{./ispalindrom -s -o example.out example.in}
      $ \osueinput{cat example.out}
      Madam is not a palindrom
      never odd or even is a palindrom
\end{osuefmtcode}

\osueguidelinesone

\end{document}
