% TeX source file
% Sysprog SS 2005
% Beispiel 3: calculator
% Martin Kirner
\input{../../template.ltx}

\begin{document}

\osuetitle{2}

\section*{Aufgabenstellung -- calc}

\begin{verbatim}
    SYNOPSIS:
        calculator
        $> <zahl1> <zahl2> <operator>

    BNF:
        <zahl>     ::= -?[0-9]+
        <operator> ::= +|-|*|/
\end{verbatim}

Implementieren Sie die Funktionalit{\"a}t eines Taschenrechners als
Elternprozess, der seinen Input
von \osueglvar{stdin} als String einliest und den String an einen
Kindprozess weiterleitet. Dieser extrahiert aus dem String die
Operanden und die Operation, berechnet das Ergebnis und gibt dieses
als String wieder zurück. Das Ergebnis wird vom Elternprozess
anschließend am Bildschirm wieder ausgegeben. Da als Zahlen
Integerwerte (maximal 65535) angenommen werden sollen, kann die
Eingabezeichenkette auf 15 Zeichen beschränkt werden.

Beispielsweise soll die Eingabe \verb_10 15 +_ das Ergebnis \verb_25_
liefern.

\subsection*{Anleitung}

Das Programm soll nach dem Starten in einer Schleife wiederholt
Rechnungen von der Tastatur (\osueglvar{stdin}) einlesen. Zur
Berechnung soll Ihr Programm einen Kindprozess erzeugen, an den
es die Eingabe über eine Pipe weiterleitet. Dieser Kindprozess
leitet nach der Umwandlung und Berechnung das Ergebnis über eine
zweite Pipe an den Elternprozess zurück. Der Elternprozess
gibt dieses Ergebnis wieder am Bildschirm (\osueglvar{stdout}) aus.

Dieser Vorgang soll solange wiederholt werden, bis der Elternprozess
\osueconst{EOF} (\osuekeystroke{Ctrl+D} von der Tastatur) liest.
In diesem Fall ist die Pipe zum Kindprozess zu schließen. Der
Kindprozess erhält dadurch beim Lesen ebenfalls \osueconst{EOF}
und terminiert. Der Elternprozess soll auf die Terminierung des
Kindprozesses warten, alle benötigten Ressourcen (Pipes) an das
System zurückgeben (schließen) und dann ebenfalls terminieren.

Sie können davon ausgehen, dass alle Rechnungen richtig eingegeben
werden -- es ist also keine Syntaxüberprüfung notwendig. Auch können
Sie die Ergebnisse abrunden (Kommastellen abschneiden) -- die Eingabe
\verb_5 2 /_ ergibt als Ergebnis daher \verb_2_ anstelle von
\verb_2.5_.

\osueguidelinestwo

\end{document}
