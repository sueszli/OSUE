% TeX source file
% Sysprog WS 2005
% Beispiel 2: chstat
% Dietmar Schabus (e0147322@student.tuwien.ac.at)
\input{../../template.ltx}

\begin{document}

\osuetitle{3}

\section*{Aufgabenstellung -- chstat}

Implementieren Sie zwei Programme \osueprog{readin} und \osueprog{chstat}, die
zu einem Eingabetext eine Buchstabenstatistik erstellen.

\begin{verbatim}
    SYNOPSIS:
        readin
        chstat [-v]
\end{verbatim}

\subsection*{Anleitung}

Das Programm \osueprog{readin} liest von \osueglvar{stdin} und übergibt jede
gelesene Zeile über ein Shared Memory an das Programm \osueprog{chstat}, welches
die Zeile liest und zu jedem Buchstaben speichert, wie oft er vorgekommen ist.
Groß-/Kleinschreibung soll dabei nicht beachtet werden, es wird also zwischen
\osueinput{A} und \osueinput{a} nicht unterschieden. Weiters können Sie davon
ausgehen, dass nur \textsc{ascii}-Zeichen vorkommen, Umlaute und dergleichen
müssen Sie also nicht behandeln.

\osueprog{chstat} gibt dann eine Liste der Buchstaben A bis Z aus, jeweils mit
der Anzahl der Vorkommnisse sowie einer prozentuellen Angabe der Häufigkeit des
Buchstabens (abgerundet auf ganze Prozent). Alle anderen Zeichen werden in einer
Kategorie "`andere"' zusammengefasst. Zusätzlich wird die Gesamtanzahl der
gelesenen Zeichen ausgegeben.

Beispiel:
\begin{verbatim}
       A: 12    15%
       B: 6     7%
       C: 8     10%
       .
       .
       .
       Z: 0     0%
  andere: 24    30%
  gesamt: 80    100%
\end{verbatim}

Wird \osueprog{chstat} mit der Option \osuearg{-v} aufgerufen, gibt es nach
jeder gelesenen Zeile die aktualisierte Statistik aus, ohne \osuearg{-v} wird
nur die endgültige Statistik über alle Zeilen ausgegeben.

\osueprog{chstat} soll die nächste Zeile erst dann lesen, wenn \osueprog{readin}
sie vollständig ins Shared Memory geschrieben hat und \osueprog{readin} soll
erst dann die nächste Zeile ins Shared Memory schreiben, wenn die vorherige
schon von \osueprog{chstat} gelesen wurde.

Durch die Eingabe von \osueconst{EOF} (also durch Drücken von
\osuekeystroke{Ctrl+D}) wird das Ende des Textes angezeigt und beide Programme
terminieren.

\subsection*{Testen}

Die Option \osuearg{-v} von \osueprog{chstat} eignet sich gut zum interaktiven
Testen, da die Auswirkungen einer neuen Zeile auf die Statistik sofort
sichtbar sind; das Testen ohne \osuearg{-v} ist für die Umleitung von
\osueglvar{stdin} auf ein Testfile vielleicht sinnvoller:
\begin{verbatim}
  readin < mytestfile
\end{verbatim}

\osueguidelinesthree

\end{document}
