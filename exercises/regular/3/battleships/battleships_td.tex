\input{../../template.ltx}

\begin{document}

\osuetitle{3}

\section*{Aufgabenstellung -- battleships}

In dieser Aufgabe werden Sie eine Terminalversion des Spiels "`Battleships"'
(auch bekannt als "`Schiffe versenken"') für 2 Spieler implementieren.

Das Spielfeld besteht aus 16 Feldern die in einem 4x4 Quadrat angeordnet sind.

Zu Beginn des Spiels müssen beide Spieler ein Schiff der Länge 3 auf seinem
Spielfeld platzieren. Das Schiff kann horizontal, vertikal oder diagonal
angeordnet werden.

Nachdem beide Spieler ihr Schiff platziert haben, beginnt das Spiel. In jeder
Runde versuchen die Spieler das Schiff ihres Gegners zu treffen. Dazu geben sie
eine Position am Spielfeld an, auf die sie schießen wollen. Nach jedem Schuss
erhalten sie eine Rückmeldung, ob das Schiff des Gegners getroffen oder
verfehlt wurde.

Ziel des Spiels ist es, das Schiff das Gegners so schnell wie möglich zu
versenken.  Ein Schiff gilt als versenkt, wenn alle 3 Felder des Schiffs
getroffen wurden.


\subsection*{Anleitung}

Das Spiel soll als Client/Server Programm realisiert werden, wobei ein Server
immer nur ein Spiel zwischen 2 Clients unterstützen muss.
Die Kommunikation zwischen Clients und Server soll per Shared Memory erfolgen.

\subsubsection*{Server}

\begin{verbatim}
  USAGE: battleships-server
\end{verbatim}

Der Server soll ein Spiel zwischen 2 Spielern verwalten können.  Sobald ein
Spiel gestartet wird sollen die benötigten Ressourcen vom Server angelegt
werden.  Bei Beendigung eines Spiels müssen diese natürlich wieder freigegeben
werden.


Zu Beginn wartet der Server bis sich 2 Spieler verbinden. Anschließend wird ein
neues Spiel gestartet. Die beiden Clients müssen dem Server die Position ihrer
Schiffe bekannt gegeben. Der Server ist für die Verwaltung der Position der
Schiffe verantwortlich.

Danach beginnt das eigentliche Spiel, und der Server verarbeitet abwechselnd
die Schüsse der Spieler. Stellen Sie sicher, dass der Server die Eingaben der
Clients abwechselnd verarbeitet, also kein Spieler mehrere Züge direkt
nacheinander durchführen kann. Nach jedem Schuss gibt der Server dem Client
bekannt, ob das gegnerische Schiff getroffen oder verfehlt wurde. Sobald ein
Spieler das Schiff des Gegners versenkt hat, teilt der Server das beiden
Clients mit und beendet das Spiel.

Anschließend wartet der Server, bis sich wieder 2 Clients verbinden.

Falls der Server ein \verb|SIGTERM| oder \verb|SIGINT| Signal erhält, sollen
alle angelegten Ressourcen (lokal angelegter Speicher, Shared Memory,
Semaphoren, etc) korrekt freigegeben und terminiert werden.

\subsubsection*{Client}

\begin{verbatim}
USAGE: battelships-client
\end{verbatim}

Der Client ist für die Darstellung des Spielfelds und die Behandlung von
Benutzereingaben zuständig.

Die Gestaltung der Oberfläche ist nicht weiter spezifiziert, außer dass
mindestens das Spielfeld und eine kurze Anleitung mit den möglichen Aktionen
angezeigt werden müssen.

Sobald der Client gestartet ist, verbindet er sich mit dem Server und wartet
bis der Server ein neues Spiel startet.

Anschließend wird der Spieler aufgefordert, die Position seines Schiffs
einzugeben.

Haben beide Spieler ihre Schiffe platziert, beginnt das Spiel.

Mögliche Aktionen sind:

\begin{itemize}
  \item \emph{Schuss auf Position am Spielfeld}: Angabe der Position am
    Spielfeld, auf die geschossen werden soll.
  \item \emph{Aufgeben}: Der Client teilt dem Server mit, dass er aufgibt
    und terminiert.  Daraufhin soll das Spiel beendet werden.
\end{itemize}

Am Spielfeld sollen die bisherigen Schüsse des Spielers angezeigt werden und ob
es sich um einen Treffer handelt.

Der Client soll zusätzlich ordnungsgemäß terminieren sobald das Spiel gewonnen
oder verloren ist, oder der Server (d.h.\ das Shared Memory oder die Semaphore)
nicht mehr erreichbar ist.

\osueguidelinesthree

\end{document}
