\documentclass{article}
\usepackage[utf8]{inputenc}
\usepackage[ngerman]{babel}
\usepackage{a4wide}
\usepackage{url}
\usepackage{hyperref}\hypersetup{colorlinks=true,linkcolor=blue,urlcolor=blue,citecolor=blue}
\parindent0pt

\begin{document}
\begin{center}
\begin{Large}
OPERATING SYSTEMS UE BONUS BEISPIEL
\end{Large}
\end{center}


\section*{Vorbereitung}
Zur Vorbereitung lesen Sie bitte die ersten drei Kapitel sowie Kapitel
5 des Buches "`Linux Device
Drivers"'~\cite{Corbet:2005:LDD:1209083}. Weiters steht Ihnen ein
rudiment{\"a}res \emph{"`Hello World"'} Modul inkl. Makefile in
der Entwicklungsumgebung zur Verf{\"u}gung. Dieses k{\"o}nnen Sie gerne als
Grundlage f{\"u}r die Entwicklung Ihres eigenen Kernel-Moduls
verwenden.

\section*{Abgabegespr{\"a}ch}
Das Abgabegespr{\"a}ch findet am 28.06.2018 im TI-Labor statt.
Eine Anmeldung in TUWEL zu einem Slot ist erforderlich -
bitte bedenken Sie, dass eine nachtr{\"a}gliche Abgabe {\bf nicht}
m{\"o}glich ist. Es gelten die
bekannten Richtlinien mit den zus{\"a}tzlichen Einschr{\"a}nkungen
bzw. Lockerungen:
\begin{itemize}
\item Ihre Implementierung muss in der TI-Lab Umgebung (User Mode Linux) demonstriert werden.
\item GNU C Standard (C99 Standard mit GNU Erweiterungen)
\item Kernel Coding Style konform (siehe $<$kernel source
  dir$>$/Documentation/CodingStyle), Coding Style Check Tool
  (auch in der Entwicklungsumgebung verf{\"u}gbar):
  \begin{verbatim}
  # /usr/src/linux/scripts/checkpatch.pl -f <source file>
  \end{verbatim}
\item Das Kernel-Modul muss sich sauber (d.h. Freigabe aller
  verwendeten Ressourcen) aus dem System per \emph{rmmod(8)} entfernen lassen.
\item Definieren Sie den Modul-Parameter \verb|debug|, der, wenn er beim Laden des Moduls auf \texttt{1} gesetzt wird, f{\"u}r sinnvolle Debugausgaben (z.B. bei den einzelnen Operationen auf den Character Devices) sorgt.
\end{itemize}

\section*{Secvault, a Secure Vault}
%TODO: je nach Aufwand koennte Verschluesselung und zusaetzliche Chardevs weggelassen werden bzw. etwas Code vorgegeben werden (+1)\\

Implementieren Sie ein Linux Kernel Modul, welches {\"u}ber Character Devices bis zu 4 fl{\"u}chtige, aber sichere Speicher (Secure Vaults - kurz: \emph{Secvaults}) zur Verf{\"u}gung stellt.
Ein \emph{Secvault} hat eine \texttt{id} und eine konfigurierbare Gr{\"o}sse (\texttt{size}) zwischen 1 Byte und 1 MByte.
Er soll per eigenem Character Device vom Userspace aus beschreib- und lesbar sein. Daten, die auf ein \emph{Secvault} Character Device geschrieben werden, sollen verschl{\"u}sselt als Cyphertext im Speicher abgelegt werden.
Beim Lesen soll der im Speicher abgelegte Cyphertext entschl{\"u}sselt werden und als Klartext in den Userpace {\"u}bergeben werden.
Als Verschl{\"u}sselung soll ein einfaches symmetrisches \emph{XOR}-Verfahren dienen: Abh{\"a}ngig von der Position im \emph{Secvault}, wird ein Schl{\"u}sselbyte mit einem Datenbyte per XOR verkn{\"u}pft:\\
$crypt(pos, data, key) = data[pos]\ \oplus\ key[pos\ mod\ key_{size}]$

Ein \emph{Secvault} hat also folgende Eigenschaften:
\begin{itemize}
\item \texttt{id}   [0--3]
\item \texttt{key}  (10 Byte)
\item \texttt{size} [1--1048576]
\end{itemize}
%TODO:

Die Verwaltung der \emph{Secvaults} soll mit Hilfe des von Ihnen zu entwickelnden Userspace-Tools \texttt{svctl} erfolgen:
\begin{verbatim}
USAGE: ./svctl [-c <size>|-k|-e|-d] <secvault id>
\end{verbatim}

Wird keine Option angebeben, soll die Gr{\"o}{\ss}e des \emph{Secvaults} in Bytes ausgegeben werden.
Die Option \texttt{-c} erzeugt einen neuen \emph{Secvault} der
Gr{\"o}{\ss}e \texttt{size} Bytes im Kernel.
Weiters sollen bei der Option \texttt{-c} von \emph{stdin} 10 Zeichen gelesen werden, die als Schl{\"u}ssel dienen.
Der Rest einer l{\"a}ngeren Eingabe wird ignoriert; bei einer k{\"u}rzeren Eingabe wird der Rest des Schl{\"u}ssels mit 0x0 gef{\"u}llt.
Die Option \texttt{-k} soll die {\"A}nderung des Schl{\"u}ssels eines \emph{Secvaults} erlauben.
Dabei soll ebenfalls wie zuvor beschrieben ein Schl{\"u}ssel von \emph{stdin} eingelesen werden, der den alten Schl{\"u}ssel ersetzt.
Der m{\"o}gliche Inhalt eines \emph{Secvaults} soll nicht ge{\"a}ndert werden.
Die Option \texttt{-e} l{\"o}scht die Daten eines existierenden \emph{Secvaults} - d.h. der gesamte Speicher wird Kernel-Modul intern mit 0x0 beschrieben (nicht indirekt {\"u}ber das \emph{Secvault} Character Device).
Die Option \texttt{-d} soll den \emph{Secvault} aus dem System entfernen und den Speicher wieder freigeben.

Das Kernel Modul legt direkt nach dem Laden (\emph{insmod(8)}) ein eigenes Character Device (ansprechbar {\"u}ber \verb|/dev/sv_ctl|\footnote{Sie m{\"u}ssen dieses Devicefile erst (einmalig) anlegen! Siehe Anleitung.} zur Steuerung und Statusabfrage an.
Das Userspace-Tool \emph{svctl} soll per \emph{IOCTL} Calls mit dem Kernel Modul kommunizieren.

\section*{Anleitung}
\begin{itemize}
\item Character (und Block) Devices werden {\"u}ber Major und Minor
  Device Numbers im Filesystem {\"u}ber \emph{Special Files}
  referenziert. Nehmen Sie 231 als Major Device Number f{\"u}r die
  \emph{Secvault} Devices. Die \emph{Special Files} k{\"o}nnen Sie per
  \emph{mknod(1)} im Verzeichnis \verb|/dev/| anlegen (es empfiehlt
  sich ein Script daf{\"u}r im Homedirectory anzulegen).
\item {\"U}ber das \emph{Secvault} Control Device (\verb|/dev/sv_ctl|)
  k{\"o}nnen per \emph{open}, \emph{release} und \emph{ioctl}
  Kommandos an das \emph{Secvault} Device abgesetzt werden.
\item Per Userspace-Tool \emph{svctl} sind mit \emph{IOCTL} Calls
  folgende Operationen m{\"o}glich:
\begin{itemize}
\item einen \emph{Secvault} anlegen und dabei Gr{\"o}sse und Schl{\"u}ssel
  festlegen
\item die Gr{\"o}{\ss}e eines \emph{Secvaults} abfragen
\item den Schl{\"u}ssel eines \emph{Secvaults} {\"a}ndern
\item einen \emph{Secvault} mit 0x0 initialisieren ( = Inhalt l{\"o}schen)
\item einen \emph{Secvault} entfernen (inkl. Speicherfreigabe)
\end{itemize}
Bei der Erstellung eines neuen \emph{Secvaults} soll auch ein neues
Character Device (ansprechbar {\"u}ber \verb|/dev/sv_data[0-3]|)
angelegt werden.
\item {\"U}ber \emph{Secvault} Data Devices (\verb|/dev/sv_data[0-3]|)
  soll der Zugriff auf den verschl{\"u}sselten Speicher per
  \emph{open}, \emph{release}, \emph{seek}, \emph{read} und
  \emph{write} erfolgen.
\item Implementieren Sie eine geeignete Behandlung von
  Fehlerf{\"a}llen:
\begin{itemize}
\item {\"u}ber Speichergr{\"o}{\ss}e des \emph{Secvaults} hinaus
  lesen/schreiben
\item Anlegen eines existierenden \emph{Secvaults}
\end{itemize}
\end{itemize}
\section*{Target und Entwicklungsumgebung}
Da die Entwicklung von Kernel-Modulen aufgrund der Systemn{\"a}he bei
Fehlern leicht zum Absturz des kompletten Systems f{\"u}hren kann,
haben wir eine Entwicklungsumgebung eingerichtet, die den
Normal-Betrieb im TI-Lab nicht st{\"o}rt: Im Verzeichnis:
\begin{verbatim}
/opt/osue/uml/
\end{verbatim}
befindet sich eine User Mode Linux (UML) Installation. Der darin
enthaltene Linux Kernel (\verb|vmlinux|) wird als normaler User
Prozess ausgef{\"u}hrt und kann auf dem Host-System auch nur
Operationen durchf{\"u}hren, die Sie als Benutzer im TI-Lab durchf{\"u}hren
k{\"o}nnen. Insbesondere ist es (in den meisten F{\"a}llen)
nicht m{\"o}glich aufgrund von Programmierfehler das Hostsystem
(TI-Lab Client bzw. den TI-Lab Application Server) zum Abst{\"u}rzen zu bringen.

Im angegebenen Verzeichnis finden Sie au{\ss}erdem das Script
\texttt{start}, welches bei Ausf{\"u}hrung eine Instanz des UML
Kernels bootet. Beim Bootprozess nutzt der UML Kernel ein
Minimal-Debian System\footnote{http://www.debian.org} (ohne
graphischer Benutzeroberfl{\"a}che) als Root-Image. Bitte ignorieren
Sie Fehlermeldungen betreffend des Netzwerkinterfaces. Am Ende des
Bootprozesses werden alle virtuellen Konsolen mit 6 Pseudoterminals am
Hostsystem verbunden. Die Ausgabe des \verb|start| Skripts bis
zum Ende des Bootvorgangs sieht typischerweise so aus:
\begin{verbatim}
$ /opt/osue/uml/start
 Using swap file: /home/b/.uml.swap
 Using copy-on-write file: /home/b/.uml.cow
...
systemd[1]: Detected virtualization 'uml'.
systemd[1]: Detected architecture 'x86-64'.

Welcome to Debian GNU/Linux 8 (jessie)!
...
[  OK  ] Reached target Login Prompts.
[  OK  ] Reached target Multi-User System.
[  OK  ] Reached target Graphical Interface.
         Starting Update UTMP about System Runlevel Changes...
[  OK  ] Started Update UTMP about System Runlevel Changes.
Virtual console 5 assigned device '/dev/pts/13'
Virtual console 4 assigned device '/dev/pts/15'
Virtual console 3 assigned device '/dev/pts/16'
Virtual console 2 assigned device '/dev/pts/18'
Virtual console 1 assigned device '/dev/pts/19'
Virtual console 6 assigned device '/dev/pts/22'
random: nonblocking pool is initialized
\end{verbatim}

{\bf Achtung:} Das Terminal, in dem Sie die UML Instanz gestartet
haben, reagiert nun auf keine Eingaben mehr.
%Weiters sollten Sie die
%Gr{\"o}sse des Terminal-Windows nicht ver{\"a}ndern.

Beachten Sie, dass diese Pseudoterminals je nach Auslastung vergeben
werden - d.h. dass Sie i.d.R. nicht immer die gleichen IDs haben. Nun
k{\"o}nnen Sie sich in einem anderen Terminalfenster beispielsweise
per \texttt{screen} mit einem der Pseudoterminals verbinden (1x Return
Taste nach dem Starten von Screen dr{\"u}cken):
\begin{verbatim}
# screen /dev/pts/25

Debian GNU/Linux 8 sysprog-bonus tty6

sysprog-bonus login: 
\end{verbatim}

Der Login lautet: \verb|root|\\
Das Passwort: \verb|rootme|\\
Im UML System haben Sie nun root-Rechte und k{\"o}nnen mit der
Entwicklung des Kernel-Moduls beginnen. Wir haben Ihnen daf{\"u}r noch
ein Script geschrieben, welches die Kernelquellen und Ihr
Homeverzeichnis in das UML-System einbindet. F{\"u}hren Sie dazu
bitte den Befehl im UML-System aus:
\begin{verbatim}
prepare <ti-lab username>
\end{verbatim}

Sie finden danach unter \texttt{/usr/src/linux/} die Kernel-Quellen
und unter \texttt{/root/homedir/} Ihr Homeverzeichnis.

\subsection*{Compilieren des Testmodules}
Hier eine kurze Session die das Compilieren, Laden und Entfernen eines
"`Hello-World"' Testmodules zeigt (angenommen wird, dass
\texttt{\~/prepare} bereits aufgerufen wurde):
\begin{verbatim}
root@sysprog-bonus:~# ls /usr/src/linux
COPYING        Makefile        block     init    modules.builtin  sound
CREDITS        Module.symvers  crypto    ipc     modules.order    tools
Documentation  README          drivers   kernel  net              usr
Kbuild         REPORTING-BUGS  firmware  lib     samples          virt
Kconfig        System.map      fs        linux   scripts          vmlinux
MAINTAINERS    arch            include   mm      security         vmlinux.o
root@sysprog-bonus:~# cd ~
root@sysprog-bonus:~# cp test_module homedir/ -R
root@sysprog-bonus:~# cd homedir/test_module
root@sysprog-bonus:~/homedir/test_module# make clean all
make V=1 ARCH=um -C /usr/src/linux M=$PWD clean;
make[1]: Entering directory '/media/host/linux-4.0.2'
[...]
mkdir -p /root/homedir/test_module/.tmp_versions ; rm -f /root/homedir/test_module/.tmp_versions/*
make -f ./scripts/Makefile.build obj=/root/homedir/test_module
[...]
make[1]: Leaving directory '/media/host/linux-4.0.2'
root@sysprog-bonus:~/homedir/test_module# insmod ./tm_main.ko
root@sysprog-bonus:~/homedir/test_module# rmmod ./tm_main.ko
root@sysprog-bonus:~/homedir/test_module# dmesg | tail
[...]
Hello World! I am a simple tm (test module)!
Bye World! tm unloading...
\end{verbatim}

\section*{Hinweise}

\begin{itemize}
\item Entwickeln Sie nach M{\"o}glichkeit auf einem TI-Lab Client
  (d.h. ein tiXX.tilab.tuwien.ac.at Rechner). Diese sind - sofern
  eingeschaltet - ebenfalls per ssh direkt von au{\ss}en erreichbar.
\item Sie k{\"o}nnen den Quelltext des Kernelmoduls auch am Host
  editieren (innerhalb ihres Homeverzeichnisses)
\item Das Terminallayout k{\"o}nnen Sie mit dem Kommando \texttt{resize}
  auf die Gr{\"o}{\ss}e des \emph{screen}-Fensters anpassen.
  Nach einem Login ist die Gr{\"o}{\ss}e auf nur 80 Spalten
  und 25 Zeilen gesetzt.
\item Die UML Instanz k{\"o}nnen Sie sauber terminieren, indem Sie im
  UML System den Befehl \texttt{halt} ausf{\"u}hren.
\item Entwickeln und Testen Sie nach M{\"o}glichkeit in kurzen Zyklen:
  Debugging, wie Sie es im Userspace per \emph{gdb} gewohnt sein
  m{\"o}gen, steht Ihnen nur mit sehr viel mehr Aufwand im Kernelmode
  zur Verf{\"u}gung. Erweitern Sie daher Ihr Modul in kleinen
  Schritten und testen gr{\"u}ndlich bereits implementierte Funktionen,
  bevor Sie darauf aufbauen.
\item Es k{\"o}nnen die Character Device Files im Filesystem (unter \verb|/dev/|)
  unabh{\"a}ngig von den tats{\"a}chlich vorhandenden Devices im
  Kernel existieren (d.h. diese m{\"u}ssen nicht direkt vom Modul
  angelegt werden bzw. wieder gel{\"o}scht werden).
\item Registrieren Sie gleich beim Laden des Moduls eine \emph{character
  device region}, in der alle f{\"u}nf Devices (ein Control- und vier
  \emph{Secvault}-Devices) Platz finden.
\item Sie k{\"o}nnen das Unixtool \emph{dd(1)} zum Testen der
  einzelnen \emph{Secvaults} verwenden. Testen Sie insbesondere Ihre
  Behandlung der m{\"o}glichen Fehlerf{\"a}lle (z.B. Schreiben/Lesen
  {\"u}ber die Grenzen des \emph{Secvaults}, ...).
\item Mit dem Kommando \emph{su(1)} k{\"o}nnen Sie die Identit{\"a}t
  von einem der drei Testuser (test1, test2 oder test3) annehmen.
%\item Falls die Meldung
%  \verb|'<module name>' likely not compiled with -mcmodel=kernel| beim
%  Laden des Moduls via insmod auftauchen sollte, bitte UML Instanz neu
%  starten.
\end{itemize}

\section*{Fragen}
\begin{itemize}
\item Ist es Ihnen m{\"o}glich den \emph{Secvault} so einzurichten,
  dass nur diejenigen Benutzer, die ihn angelegt haben, darauf
  schreiben und davon lesen k{\"o}nnen? Jeder Benutzer des Systems
  soll, sofern der angeforderte \emph{Secvault} frei ist, in der Lage
  sein, einen \emph{Secvault} anzulegen.\\
  {\bf Bemerkung: bis zu +5 weitere Bonuspunkte bei korrekter
    Implementierung.}
\item Wie verhindern Sie unsynchronisierten Zugriff bei
  "`gleichzeitiger"' Verwendung des selben \emph{Secvault} Data
  Devices ?
\item Angenommen, ein Angreifer m{\"o}chte den unverschl{\"u}sselten
  Inhalt eines \emph{Secvaults} kennen und hat die M{\"o}glichkeit RAM direkt
  auszulesen\footnote{
    Die Nintendo 3DS Spielekonsole nutzte beispielsweise einen externen
    RAM Bauteil, wo Daten abgelegt wurden. Ein Angriff auf die CPU Kerne
    in einem System-on-Chip (SoC) Package sowie auf den Programmcode war
    aus diversen Gr{\"u}nden unpraktisch. Die Daten in diesem externen
  RAM Bauteil waren jedoch abgreifbar. Siehe:
  https://owncloud.tuwien.ac.at/public.php?service=files\&t=401f7fe05e214ea6acaeeffabd1224d7
(by Neimod)}. Andere Systemkomponenten, insb. der \emph{Secvault} Schl{\"u}ssel,
CPU und Programmcode, sind f{\"u}r den Angreifer jedoch nicht einsehbar
oder sogar beeinflu{\ss}bar. Ist der Angreifer in der Lage den Plaintext
des \emph{Secvaults} wiederherzustellen? Welche maximale Secvaultgr{\"o}{\ss}e
w{\"a}re Ihrer Meinung nach sicher oder ist jede Gr{\"o}{\ss}e sicher/unsicher?
\end{itemize}

\bibliographystyle{plain}
\begin{thebibliography}{9}

\bibitem{Corbet:2005:LDD:1209083}
    Jonathan Corbet, Alessandro Rubini, Greg Kroah-Hartman
    \emph{Linux Device Drivers},
    O'Reilly Media, Inc.,
    3rd Edition,
    2005,
    0596005903,
    Verf{\"u}gbar unter: \url{http://lwn.net/Kernel/LDD3/}.

\end{thebibliography}

\end{document}

